\documentclass[a4paper,10pt]{article}
\usepackage[utf8]{inputenc}
\usepackage[spanish]{babel}
%links en el indice
\usepackage[bookmarks = true, colorlinks=true, linkcolor = black, citecolor = black, menucolor = black, urlcolor = black]{hyperref} 
\usepackage{graphicx}



%opening
\title{Triage, Sistema de gestión para sala de Guardia Hospitalaria}
\author{Néstor Muñoz\\ Marcia Tejeda\\ \\ Director: Nicolás Paez \\  Co-Director: Pablo E. Martínez López}



\begin{document}



\maketitle
\newpage
\newpage 
\begin{abstract}
El presente Trabajo de Inserción Profesional (TIP) fue realizado en el contexto del desarrollo de un sistema para resolver necesidades de la guardia del Hospital Oñativia en Rafael Calzada.\\ 
El problema presentado en este TIP es automatizar la recepción en la guardia del Hospital utilizando el método de emergencias conocido como Triage.
 
La solución propuesta es el desarrollo de una aplicación web que sea accesible desde todos los puntos de atención de las guardias y que permita evaluar los casos que se presentan de manera eficiente y con los mismos parámetros.

\end{abstract}


\newpage 
\tableofcontents

\newpage


\section{Introducción}

Actualmente la guardia del H.Z.G.A ``Dr.\ Arturo Oñativia'' de la localidad de Rafael Calzada, a cargo del Doctor Luis Reggiani, utiliza el método Triage \cite{Derlet,Manual} para la clasificación de pacientes según los síntomas que presenten. 
Triage es un método de la medicina de emergencias y desastres para la selección y clasificación de los pacientes basándose en las prioridades de atención, privilegiando la posibilidad de supervivencia, de acuerdo a las necesidades terapéuticas y los recursos disponibles. Trata de evitar el agravamiento del diagnóstico del paciente a causa de demora en la atención. Un nivel que implique que el paciente puede ser demorado no quiere decir que el diagnóstico final no pueda ser una enfermedad grave, ya que un cáncer, por ejemplo, puede tener funciones vitales estables que no lleve a ser visto con premura. El triage prioriza el compromiso vital inmediato y las posibles complicaciones.

Hasta el momento todo el proceso de Triage en el hospital se hace en forma manual, lo que implica algunos contratiempos:
\begin{itemize}
\item Depender de una persona (o varias) con todo el conocimiento.
\item Emplear demasiado tiempo para guardar datos y recolectarlos.
\item Obtener diferentes resultados (algunas veces incorrectos), pues diferentes personas usan en ocasiones criterios diferentes para la toma de decisiones.
\end{itemize}

Según el Dr.\ Reggiani informatizar el proceso de Triage implicaría una mejora notable en el desempeño de la guardia. Se lograría una estandarización en la clasificación de síntomas, se agilizaría el ingreso y la obtención de datos de pacientes, se mejoraría la atención en general y se distinguirían de una manera más eficaz aquellos pacientes que necesiten una atención inmediata.



En este trabajo proponemos desarrollar y poner en funcionamiento un sistema informático que dé soporte al proceso de Triage en la guardia del H.Z.G.A ``Dr.\ Arturo Oñativia'' de la localidad de Rafael Calzada. Dado que el sistema podría incluir muchísimas funcionalidades y al mismo tiempo existe una especificación detallada de los requerimientos, planteamos el proyecto con alcance variable con el compromiso de entrega de un software que resuelva la parte central del proceso de Triage. La idea es que el sistema desarrollado en el contexto de este trabajo sea puesto en marcha y utilizado por la institución promotora.

Dado el contexto en el cual debemos realizar el proyecto, consideramos que lo más apropiado es el uso de una metodología ágil \cite{Shore}. En este sentido trabajamos con iteraciones de tiempo fijo de una semana de duración y cada incremento del sistema es validado por el Dr.\ Reggiani quien ocupa simultáneamente los roles de responsable de producto y especialista de negocio.

Concretamente el sistema debe cubrir las siguientes funcionalidades mínimas:
\begin{itemize}
\item Recepción de pacientes mediante búsqueda de aquellos que ya fueron atendidos en el hospital e ingreso de los que se atienden por primera vez.
\item Toma de impresión visual inicial del paciente.
\item Toma de los signos vitales que presenta el paciente: presión arterial (sístole y diástole), frecuencia cardíaca, saturación de O2, frecuencia respiratoria, temperatura y glucosa.
\item Ingreso de los síntomas que presenta el paciente.
\item División de los síntomas por categorías (discriminantes) y asociación de prioridades a los mismos.
\item Lógica variada para los síntomas, según se trate de un paciente adulto o pediátrico, tanto para los valores de los signos vitales como para las prioridades de los síntomas.
\item Emisión de alerta al momento de detectarse un síntoma de prioridad uno, para que se ingrese al paciente de inmediato al shock room.
\item Posibilidad de extraer reportes de cantidad de consultas realizadas según prioridad y promedio de tiempo de espera de atención según prioridad.
\item Puesta en funcionamiento en cada sala de recepción de pacientes de guardia.
\end{itemize}

Este informe cuenta nuestro trabajo en el desarrollo, implementación y puesta en funcionamiento de una aplicación web que cubre todas las funcionalidades mínimas detalladas anteriormente y además realiza las siguientes:
\begin{itemize}
\item Generación de reportes por paciente a modo de historial de atenciones en guardia con detalle de fecha, síntomas presentados, signos vitales, prioridad asignada y tipo de atención recibida.
\item Diferenciación entre usuarios administradores del sistema y usuarios comunes.
\item Posibilidad de detallar el tipo de atención recibida por el paciente luego de pasar por el proceso de Triage
\item Alta, baja y modificación de pacientes, síntomas, discriminantes de síntomas y usuarios del sistema.
\end{itemize}


\subsection{Sumario}
En la sección 2 de este documento hablaremos del planteo del problema: cómo es la dinámica de trabajo en el hospital, cuáles son los requerimientos del cliente y cómo proponemos que sea la aplicación resultante de este trabajo. Luego, en la sección 3 describiremos todas las herramientas tecnológicas utilizadas para el desarrollo así como las metodologías de trabajo implementadas y ampliaremos los detalles de diseño e implementación. En la sección 4, detallaremos las pruebas realizadas y explicaremos cómo procedimos a la instalación del sistema. Para finalizar, en la sección 5 presentaremos las conclusiones. 




\section{Planteo}
\subsection{Problema en detalle}
Cómo se trabaja en el hospital, cuál es la necesidad real
\subsection{Reuniones y contacto con el usuario}
Contacto constante (Esto aparece abajo también, no sé dónde iría o cómo dividirlo)
\subsection{Requerimientos del cliente}
?
\subsection{Pantallas}
Contamos en general cuál sería la dinámica de uso y las pantallas principales
\subsection{Informes}
Qué informes nos pidió el cliente y para qué
\subsubsection{Tiempo de espera para cada prioridad}
Consultar a Luis para qué necesita este reporte
\subsubsection{Cantidad de atenciones para cada prioridad}
Consultar a Luis para qué necesita este reporte
\subsubsection{Reporte de Personas}
Lista con todas las personas que se atendieron y el detalle de cada atención. Preguntar a Luis por qué es útil


\section{Diseño e implementación}
\subsection{Tecnologías}
Todas las tecnologías que utilizamos en el desarrollo de la aplicación son de código abierto. Para desarrollar el front-end (interfaz de usuario) elegimos AngularJS\footnote{https://angularjs.org/} y para el back-end (lógica del negocio e interacción con la base de datos) elegimos Grails\footnote{https://grails.org/}.\\
Es pertinente aclarar que ambos frameworks cubrieron ampliamente todo lo que nosotros necesitabamos de ellos, por lo tanto sólo utilizamos una pequeña parte de los mismos. De hecho, de Grails sólo utilizamos la parte del lado del servidor, ya que del lado del cliente utilizamos AngularJS.
\subsubsection{AngularJS}
AngularJS es un framework de aplicaciones web de código abierto escrito en JavaScript. Es desarrollado y mantenido por Google. La primer versión fue lanzada en el año 2010 y desde ese momento viene ganando espacio en el mercado.\\
Para desarrollar páginas web estáticas HTML es suficiente. Pero no alcanza para desarrollar vistas dinámicas en aplicaciones web. AngularJS es un conjunto de herramientas para la creación de aplicaciones web de una sola página (single-page applications) con contenido dinámico que permite extender el vocabulario HTML para obtener un entorno más expresivo, legible y práctico. La filosofía de AngularJS es que la programación declarativa es la que debe utilizarse para generar interfaces de usuario.\\
Si bien nosotros no conociamos AngularJS, habiamos escuchado buenos comentarios sobre este framework moderno y novedoso, entonces decidimos aprenderlo utilizandolo en este TIP.
\subsubsection{Grails}
Grails es un framework de aplicaciones web de código abierto, full stack (contiene todo lo necesario para desarrollar una aplicación web), para la máquina virtual de Java (JVM).\\
Está desarrollado en Groovy, un lenguaje de programación que a su vez está desarrollado en Java. Uno de sus principios es la convención sobre la configuración (convention over configuration) que busca decrementar el número de decisiones que un desarrollador necesita tomar, ganando así en simplicidad pero no perdiendo flexibilidad por ello. La primer versión fue lanzada en el año 2006.\\
Como está desarrollado en Java, Grails es multiplataforma. Además reutiliza tecnologías muy probadas en la industria como Hibernate y Spring.\\
Al igual que con AngularJS, nosotros no conociamos Grails de antemano. Tuvimos que aprender a utilizarlo en el desarrollo de este TIP.
\subsubsection{Bases de datos}
Durante todo el desarrollo de la aplicación utilizamos una base de datos H2 embebida en Grails. También utilizamos H2 en la instalación de prueba en el hospital. En cambio para la instalación definitiva utilizamos Postgres\footnote{http://www.postgresql.org.es/}.\\
Elegimos Postgres porque es lo que recomienda Heroku\footnote{https://www.heroku.com/} para utilizar con Grails\footnote{https://devcenter.heroku.com/articles/getting-started-with-grails}. Heroku es una plataforma de la nube que soporta varios lenguajes de programación y ofrece un servicio de hosting básico gratuito. Nuestra idea original era instalar la aplicación en dicha plataforma, pero el límite de memoria que ofrece el servicio gratuito nos obligó a buscar otra alternativa. Finalmente decidimos hacer la instalación en una máquina del hospital.
\subsection{Metodología de trabajo}
Desde el primer momento el director del TIP nos indujo a darle al desarrollo un enfoque ágil\cite{Shore}. Así dar visibilidad constante a todos los interesados fue uno de los principios transversales a todo el proyecto. La comunicación fue muy fluida, tanto por mail, como a través de reuniones presenciales o virtuales (en forma remota). Otro de los pilares del enfoque ágil fue trabajar en forma iterativa e incremental. Es decir que trabajamos con iteraciones de tiempo fijo de una semana de duración. Al final de cada iteración los avances eran validados por el ''cliente''.
\subsubsection{Resumen del itinerario del proyecto}
Los primero que hicimos fueron varias reuniones entre todos los interesados en el proyecto: los desarrolladores, los directores y el ''cliente''. De esas reuniones y de una visita al hospital obtuvimos los requerimientos.\\
El segundo paso fue la elección de las tecnologías. Entre el basto abanico de posibilidades NodeJS\footnote{http://nodejs.org/} y Grails aparecian como las predilectas aunque nunca habiamos trabajado con ninguna de las dos. Por ello, para obtener una impresión general de cada una y así resolver la elección, desarrollamos un conversor de Farenheit a Celcius muy simple. Luego de eso terminamos eligiendo Grails ya que se asemeja más que NodeJS a las tecnologías que veniamos utilizando en las distintas materias a lo largo de carrera.\\
Si bien Grails es full stack, el director del TIP quería darle una impronta moderna a la aplicación entonces nos recomendó utilizar una tecnología que resuelva las vistas del lado del cliente, se comunique con el back-end mediante una API(Application Programming Interface) REST\footnote{http://es.wikipedia.org/wiki/Representational\_State\_Transfer} 
y sea responsive\footnote{http://es.wikipedia.org/wiki/Diseño\_web\_adaptable}. Así surgio la idea de utilizar AngularJS que cubre ampliamente todos esos requisitos.\\
En tercer lugar hicimos una estimación relativa a grandes rasgos sobre cuanto tiempo nos iba a demandar cada funcionalidad requerida y la fecha de cierre del proyecto. También hicimos una planificación en donde ordenamos los requerimientos dentro de las iteraciones según las prioridades del ''cliente''.\\
A partir de ahí comenzamos con el desarrollo a través de las iteraciones planificadas. Al promediar el proyecto hicimos una instalación de prueba en el hospital para que los usuarios vayan interactuando con el producto, nos den un feedback y propongan modificaciones de creerlo necesario.\\
Por último, luego de finalizar el desarrollo hicimos la instalación definitiva en una máquina del hospital.
\subsubsection{Flujo de trabajo en una iteración}
Al inicio de cada iteración estimabamos cuanto tiempo nos iba a llevar cada tarea y enviabamos un email con los detalles sobre lo que ibamos a hacer durante esa semana. También haciamos prototipos de las pantallas a realizar que eran validados por el ''cliente''. Además dejamos sentado en una hoja de cálculo los detalles de cada tarea: el tiempo de realización estimado, la fecha de realización y el tiempo real insumido. Finalmente enviabamos otro email con los detalles de lo realizado en la semana.
\subsubsection{Herramientas que utilizamos}
Para la comunicación via email creamos un grupo en Google Groups\footnote{https://groups.google.com}.
Para toda la documentación compartida utilizamos Google Drive\footnote{https://drive.google.com/}. Para hacer reuniones remotas utilizamos Skype\footnote{http://www.skype.com.ar/es/} y Google Hangouts\footnote{https://plus.google.com/hangouts}. Utilizamos Git\footnote{http://git-scm.com/} y Github\footnote{https://github.com/} para versionar el código. Y utilizamos Travis\footnote{https://travis-ci.org/} como servidor de integración continua\footnote{http://es.wikipedia.org/wiki/Integración\_continua}.
\subsection{Arquitectura}
División entre angular y grails
 
\subsection{Diseño}
Diagrama de clases
\subsection{Detalle técnico}
?

\subsection{Problemas que tuvimos}
?

\subsection{Detalles interesantes del código}
En el caso de que los haya 




\subsection{Testing}
Herramientas que usamos para testear, qué tipo de testing hicimos (funcional, unitario, de integración...)
\subsubsection{Funcional}
Selenium.. / Protractor / Casper
\subsubsection{Unitario}
Con Grails
\subsubsection{Integración}
Con Grails
\subsection{Deploy}
La idea original fue tener la aplicaciòn on line en Heroku
\subsubsection{Instalación}
Cómo realizamos la instalación en el entorno de trabajo donde se usará el producto
\subsubsection{Manual de usuario}
Se confeccionó un manual de usuario...
\subsubsection{Aprendizaje del usuario}
Cuánto tiempo llevó explicar el sistema, si fue fácil de entender..


\section{Pruebas e instalación}
En esta sección describimos todos los tipos de pruebas que realizamos junto con el desarrollo de la aplicación: pruebas unitarias, de integración y funcionales. En segundo lugar mencionamos algunos problemas con los que nos encontramos al incorporar las pruebas al desarrollo. Luego describimos cómo realizamos la instalación del producto final. Y por último mostramos algunas métricas del proyecto.

\subsection{Pruebas}
Desarrollamos la aplicación realizando pruebas unitarias y de integración de cada clase del dominio así como también pruebas funcionales de cada pantalla.

Cada funcionalidad desarrollada tiene su conjunto de pruebas correspondiente. Algunas de las ventajas de usar pruebas automatizadas son las siguientes:

\begin{itemize}
\item Se robustece la aplicación.
\item Se genera confianza en el programador al momento de hacer modificaciones.
\item Se ahorra tiempo.
\item Se disminuye el margen de error en el código.
\end{itemize}

\subsubsection{Pruebas unitarias}
En programación, una prueba unitaria es una forma de comprobar el correcto funcionamiento de un módulo de código. Esto sirve para asegurar que cada uno de los módulos funcione correctamente por separado. Luego, con las Pruebas de Integración, se podrá asegurar el correcto funcionamiento del sistema o subsistema en cuestión. Cabe mencionar que las pruebas unitarias no tienen repercusión en la base de datos. Para realizarlas utilizamos la herramienta nativa de Grails\footnote{\texttt{http://grails.org/doc/latest/guide/testing.html\#unitTesting}} con la biblioteca de Java JUnit\footnote{\texttt{http://junit.org/}}. Así cubrimos el comportamiento de las clases del dominio definidas en el \textit{back-end}

\subsubsection{Pruebas de integración}
Las pruebas de integración son aquellas que se realizan en el ámbito del desarrollo de software una vez que se han aprobado las pruebas unitarias. Se refieren a las pruebas de todos los elementos unitarios que componen un proceso, hechas en conjunto, de una sola vez. Consiste en realizar pruebas para verificar que un gran conjunto de partes de software funciona bien. Las pruebas de integración preceden a las pruebas funcionales del sistema. Cabe mencionar que este tipo de pruebas tiene repercución en la base de datos. Para realizarlas utilizamos la herramienta nativa de Grails\footnote{\texttt{http://grails.org/doc/latest/guide/testing.html\#integrationTesting}}. Con ello cubrimos el comportamiento de los controladores definidos en el \textit{back-end}.

\subsubsection{Pruebas funcionales}
Las pruebas funcionales se basan en la ejecución, revisión y retroalimentación de las funcionalidades previamente diseñadas para el software. Se hacen mediante el diseño de modelos de prueba que buscan evaluar cada una de las opciones con las que cuenta el paquete informático. Dicho de otro modo son pruebas específicas, concretas y exhaustivas para probar y validar que el software hace lo que debe y sobre todo, lo que se ha especificado. Para realizarlas utilizamos CasperJS\footnote{\texttt{http://casperjs.org/}} y Protractor\footnote{\texttt{https://github.com/angular/protractor}} haciendo una simulación del usuario final utilizando la aplicación.

En un primer momento utilizamos CasperJS pero tuvimos muchos problemas para hacerlo funcionar correctamente con AngularJS. Por eso dejamos de usarlo y lo reemplazamos por Protractor.

Para utilizar Protractor necesitamos usar Selenium\footnote{\texttt{http://www.seleniumhq.org/}}, un entorno de pruebas de software para aplicaciones basadas en la web, con un \textit{driver\footnote{\texttt{http://es.wikipedia.org/wiki/Manejador\_de\_dispositivo}}} para el navegador  Chrome\footnote{\texttt{http://www.google.com/intl/es-419/chrome/}}.

\subsubsection{Problemas que tuvimos con el desarrollo de las pruebas}
A continuación detallamos los problemas con los que nos encontramos al utilizar pruebas automatizadas.
\begin{itemize}
\item Hay algunas funciones de Grails que no son soportadas por su ambiente de pruebas. Por eso para que las pruebas pasen, nos vimos obligados a usar solo aquellas funciones que no tenían dicho problema.
\item Tuvimos problemas con Protractor al probar pantallas con ventanas modales. Las ventanas modales son elementos que al aparecer, bloquean la ventana principal de la aplicación. Por eso para poder probar estas pantallas nos vimos obligados a dormir la ejecución de la prueba durante un segundo. Así el modal tenía tiempo para desaparecer y la ventana principal de desbloquearse. Si no hacíamos esto, la prueba fallaba ya que la ejecución de la misma, luego de cerrar el modal, intentaba interactuar con elementos de la ventana principal que aún permanecían bloqueados.
\end{itemize}

\subsection{Instalación}
Acordamos con el Dr. Reggiani hacer la instalación en una máquina del hospital. Como las computadoras están en una misma red entonces la aplicación se encuentra accesible desde cualquier punto del lugar.

La primer instalación (de prueba) la hicimos al promediar el proyecto. El objetivo de la misma fue que los usuarios finales se familiarizacen con el producto, nos dieran \textit{feedback} y propusieran modificaciones de creerlo necesario. La segunda instalación (definitiva) la hicimos al finalizar el proyecto.

Para la instalación de prueba usamos un servidor Tomcat\footnote{\texttt{http://tomcat.apache.org/}} al cual le insertamos el WAR\footnote{\texttt{http://es.wikipedia.org/wiki/WAR\_(archivo)}}(archivo ejecutable que realiza la instalación del producto) de nuestra aplicación que contenía a su vez una base de datos H2\footnote{\texttt{http://www.h2database.com}} embebida.

La instalación final fue similar a la de prueba con la salvedad que no utilizamos la base de datos H2 embebida en Grails. En su lugar instalamos una base de datos PostgreSQL independiente del resto del sistema y, por lo tanto, más segura y confiable.


\subsection{Métricas del proyecto}

\begin{description}
\item[Cantidad de historias de usuario/funcionalidades implementadas] \mbox{} \\
Implementamos 16 funcionalidades.

\item[Cantidad de iteraciones de trabajo] \mbox{} \\
Realizamos el proyecto en 18 iteraciones de una semana de duración cada una. En cada iteración planificamos trabajar 10 horas por desarrollador.

\item[Cantidad de horas trabajadas] \mbox{} \\
Para realizar este proyecto dedicamos un total de 402 horas. Es decir que cada desarrollador trabajó 201 horas.

\item[Cantidad de commits en el repositorio] \mbox{} \\
Realizamos 292 commits en el repositorio de GitHub.

\item[Cantidad de pruebas unitarias] \mbox{} \\
Realizamos 20 pruebas unitarias sobre las clases del dominio Persona y Paciente.

\item[Cantidad de pruebas de integración] \mbox{} \\
Realizamos 26 pruebas de integración sobre los controladores en el \textit{backend} de Persona, Paciente y Síntoma.

\item[Cantidad de pruebas funcionales] \mbox{} \\
Realizamos 66 pruebas funcionales sobre las diferentes pantallas de la aplicación.

\end{description}


\section{Conclusiones}

\paragraph{}
El presente trabajo abordó el planteo, diseño, implementación, desarrollo y puesta en producción de ``Triage, Sistema de gestión para 
sala de Guardia Hospitalaria''





\newpage 
\begin{thebibliography}{3} 
\bibitem{Derlet} Derlet R, Kinser D, Lou R, et al. Prospective identification and triage of nonemergency patients out of an Emergency Department: a 5 years study. Ann Emerg Med 1996; 25:215-223.
\bibitem{Manual} Manual de procedimiento. Recepción,  Acogida y Clasificación.  MSPBS. Paraguay 2011.
\bibitem{Shore} Shore J, Warden S, The Art of Agile Development, O’Reilly Media, 2007.

\end{thebibliography}

\newpage
\appendix
\section{Manual del Usuario}
Las siguientes secciones tienen como objetivo describir y explicar las funcionalidades del Sistema de gestión para sala de Guardia Hospitalaria, que utiliza el método Triage para la recepción de los pacientes. En las mismas, se detalla la funcionalidad de cada pantalla con screenshot y ejemplos básicos y funcionales al manual.

Triage es un método de medicina de emergencias y desastres para la selección y clasificación de los pacientes basándose en las prioridades de atención. La guardia del H.Z.G.A ``Dr.\ Arturo Oñativia'' de la localidad de Rafael Calzada utiliza este sistema para clasificar a sus pacientes. El triage prioriza el compromiso vital inmediato y las posibles complicaciones.
Los pacientes pueden ser clasificados con tres Prioridades:

\begin{description}
\item[Prioridad 1] \mbox{} \\ 
Cuando el paciente tiene posibilidad de sobrevivir y la actuación médica debe ser inmediata.
\item[Prioridad 2] \mbox{} \\ 
Pacientes que presentan una situación de urgencia con riesgo vital.
\item[Prioridad 3] \mbox{} \\ 
Paciente levemente lesionado, que puede caminar y su traslado no precisa medio especial.

\end{description}



\mySection{Pantalla Inicial}
En esta sección del manual explicaremos cómo ingresar nuevas personas al sistema o cómo buscarlas en el caso de que ya hayan sido atendidas.

Ingresar una nueva persona a la aplicación significa que sus datos quedarán guardados en el sistema, para poder realizar consultas sobre las atenciones recibidas o bien, para no tener que volver a cargar los datos si el paciente vuelve a atenderse en otro momento.

\mySubSection{Ingreso de un nuevo paciente}
En la pantalla de Inicio del sistema (figura \ref{fig:inicio}) 
\begin{figure}
\centerline{\includegraphics[width=0.99\textwidth]{inicio.png}}
\caption{Pantalla inicial} \label{fig:inicio}
\end{figure}
se pueden ver los campos identitificatorios de las personas. El botón ``Ingresar nuevo paciente'' aparecerá deshabilitado hasta completar los campos obligatorios: DNI, nombre, apellido y fecha de nacimiento (como puede verse en la figura \ref{fig:inicio_nuevo}).
\begin{figure}
\centerline{\includegraphics[width=0.99\textwidth]{inicio_nuevo.png}}
\caption{Botón habilitado para poder cargar nuevo paciente} \label{fig:inicio_nuevo}
\end{figure}
Al presionar dicho botón, el sistema guardará los datos de esa persona y redigirá la aplicación a la ventana de Triage para comenzar con la carga de síntomas.

\mySubSection{Búsqueda e ingreso de un paciente cargado en sistema}
En el caso de que el paciente ya haya recibido atención en la guardia, es posible buscarlo en la aplicación. El botón ``buscar'' apacerá deshabilitado hasta que se complete alguno de los campos (figura \ref{fig:inicio_busqueda}).
\begin{figure}
\centerline{\includegraphics[width=0.99\textwidth]{inicio_busqueda.png}}
\caption{Botón habilitado para poder buscar un paciente y botón para ingresar al paciente a Triage} \label{fig:inicio_busqueda}
\end{figure}
Una vez presionado el botón para buscar, el sistema muestra en el listado inferior la lista de personas que coinciden con los criterios de búsqueda ingresados (ver sección \ref{cap:filtrado_listado}). En el caso de ver a la persona que se está buscando, en la lista aparece el botón ``Ingresar''. Al presionarlo, el sistema redirige la aplicación a la ventana de Triage para comenzar con la carga de síntomas.

\mySubSection{Filtrado de un listado}\label{cap:filtrado_listado}
Todos los listados de la aplicación pueden ser filtrados para facilitar la búsqueda de algún registro. El modo de filtrado es muy sencillo. Los pasos a seguir son los siguientes:
\begin{enumerate}
\item ingresamos algun texto en el/los campo/s de búsqueda (no hace falta que ingresemos la palabra entera de lo que buscamos, es suficiente si solo ingresamos las primeras letras).
\item presionamos el boton ``Buscar''.
\end{enumerate}



\section{Triage}
En esta sección daremos a conocer el camino que recorre la aplicación para cargar los síntomas del paciente.

\subsection{Pantalla inicial de Triage}
La pantalla inicial de Triage (como podemos ver en la figura \ref{fig:triage_inicial}) 
\begin{figure}
\centerline{\includegraphics[width=0.99\textwidth]{impresion_visual.png}}
\caption{Pantalla inicial de Triage} \label{fig:triage_inicial}
\end{figure}
tiene una navegación definida por pestañas (que se pueden ver en la parte izquierda de la pantalla). Las pestañas permiten cambiar de pantalla de manera rápida y simple.

\subsection{Impresión Visual}
En la pestaña de impresión visual (figura \ref{fig:triage_inicial}) se cargan los síntomas que el enfermero/administrativo ve en el paciente que está siendo atendido. Una vez seleccionados los síntomas visuales, el usuario debe presionar el botón ``Aceptar''. En el caso de que un síntoma de Prioridad UNO sea seleccionado el sistema corta la interacción con el usuario con un cartel de confirmación (figura \ref{fig:impresion_visual_p1}).
\begin{figure}
\centerline{\includegraphics[width=0.99\textwidth]{impresion_visual_p1.png}}
\caption{Pantalla inicial de Triage} \label{fig:impresion_visual_p1}
\end{figure}
Si el usuario confirma, el sistema deriva directamente a la pantalla de Prioridad UNO, mostrando los datos y síntomas del paciente ingresado (figura \ref{fig:prioridad_uno}).
\begin{figure}
\centerline{\includegraphics[width=0.99\textwidth]{prioridad_uno.png}}
\caption{Prioridad UNO} \label{fig:prioridad_uno}
\end{figure}

\subsection{Síntomas}
En la pestaña de síntomas (figura \ref{fig:sintomas})
\begin{figure}
\centerline{\includegraphics[width=0.99\textwidth]{sintomas.png}}
\caption{Pestaña de síntomas} \label{fig:sintomas}
\end{figure}
van a ser cargados los síntomas que el paciente informe. 

En el cuadro central se pueden ver todos los síntomas cargados en sistema, indicando cuál es su discriminante. Aquí se puede filtrar también por síntoma o discriminante (tal como se explica en la sección 'Filtrado del listado') (Ver figura \ref{fig:sintomas_filtrar}).
\begin{figure}
\centerline{\includegraphics[width=0.99\textwidth]{sintomas_buscar.png}}
\caption{Filtrado en el cuadro de síntomas} \label{fig:sintomas_filtrar}
\end{figure}
Una vez filtrado el listado y encontrado lo que se busca, en cada fila del cuadro se puede ver el botón ``Agregar'' (figura \ref{fig:sintomas_agregar}), que permite cargar un nuevo síntoma al paciente.
\begin{figure}
\centerline{\includegraphics[width=0.99\textwidth]{sintomas_agregar.png}}
\caption{Agregar nuevo síntoma} \label{fig:sintomas_agregar}
\end{figure}
En la parte derecha de la pantalla se pueden ver los síntomas ya cargados. Se puede también eliminar algún síntoma agregado mediante el botón ``Borrar'' (que aparece al pararse con el puntero sobre el elemento a eliminar).

En el caso de ingresar un síntoma con Prioridad UNO,  el sistema corta la interacción con el usuario con un cartel de confirmación. Si el usuario confirma que efectivamente ese es el síntoma a agregar, el sistema deriva directamente a la pantalla de Prioridad UNO, mostrando los datos y síntomas del paciente ingresado (figura \ref{fig:prioridad_uno}).

Al finalizar con la carga, se debe presionar el botón ``Aceptar'' para grabar los síntomas seleccionados.


\subsection{Signos Vitales}
La tercer pestaña del Triage es para completar los signos vitales del paciente (figura \ref{fig:signos_vitales}).
\begin{figure}
\centerline{\includegraphics[width=0.99\textwidth]{signos_vitales.png}}
\caption{Signos Vitales} \label{fig:signos_vitales}
\end{figure}
Cada signo vital está definido para ser seleccionado de una lista acotada. En el caso de seleccionar algún valor que corresponda a una Prioridad UNO, el sistema mostrará un mensaje de confirmación. Si el usuario confirma la acción, se corta toda interacción mostrando la pantalla que indica atención inmediata (figura \ref{fig:prioridad_uno}).

Al finalizar de cargar los signos vitales, se debe presionar el botón ``Guardar'' (figura \ref{fig:signos_vitales_guardar})
\begin{figure}
\centerline{\includegraphics[width=0.99\textwidth]{signos_vitales_guardar.png}}
\caption{Signos Vitales} \label{fig:signos_vitales_guardar}
\end{figure}
y el sistema informará que los datos se han guardado con éxito.


\subsection{Fin de la carga}
Al terminar de cargar los síntomas hay dos caminos:
\begin{description}
\item[Finalizar Triage]  \mbox{} \\
Para finalizar el Triage, se debe presionar el botón sobre la pestaña izquieda llamado ``Fin Triage'' (figura \ref{fin_triage}).
\begin{figure}
\centerline{\includegraphics[width=0.99\textwidth]{fin_triage.png}}
\caption{Signos Vitales} \label{fig:fin_triage}
\end{figure}
 Al hacer esto, el sistema calcula la prioridad del paciente, indicando todos los síntomas cargados y sus datos personales. 

Esta acción sólo puede mostrar las pantallas de Prioridad DOS (figura \ref{fig:prioridad_dos}) 
\begin{figure}
\centerline{\includegraphics[width=0.99\textwidth]{prioridad_dos.png}}
\caption{Prioridad DOS} \label{fig:prioridad_dos}
\end{figure}
y Prioridad TRES (figura \ref{fig:prioridad_tres}), 
\begin{figure}
\centerline{\includegraphics[width=0.99\textwidth]{prioridad_tres.png}}
\caption{Signos Vitales} \label{fig:prioridad_tres}
\end{figure}
ya que la pantalla de Prioridad UNO sólo se presenta al seleccionar un síntoma de Prioridad UNO y corta toda interacción con el usuario.
Finalizar el Triage no quita al paciente de la lista de espera, simplemente calcula su prioridad. 

\item[Salir de la carga]\mbox{} \\
En el caso de querer abandonar la carga de síntomas para poder retomarla más tarde, el sistema provee la acción ``Salir'' (figura \ref{fin}),
\begin{figure}
\centerline{\includegraphics[width=0.99\textwidth]{fin.png}}
\caption{Signos Vitales} \label{fig:fin}
\end{figure}
que permite guardar los síntomas ingresados hasta el momento y poder recuperarlos si se carga el paciente desde la lista de espera.


\end{description}

\section{Alta, baja y modificación de datos (ABMs)}
En esta sección explicamos como realizar el alta, baja y modificación de datos del sistema, es decir, síntomas, discriminantes y usuarios.

\subsection{Acceso al menú de configuración}
Para poder acceder al menú de configuración de datos el usuario actual debe tener el rol de administrador. En caso contrario dicho menú permanecerá oculto (ver figuras \ref{fig:menu_conf_visible} y \ref{fig:menu_conf_oculto}).

\begin{figure}
\centerline{\includegraphics[width=0.7\textwidth]{menu_configuracion_visible.png}}
\caption{Menú de configuración visible}
\label{fig:menu_conf_visible}
\end{figure}

\begin{figure}
\centerline{\includegraphics[width=0.7\textwidth]{menu_configuracion_oculto.png}}
\caption{Menú de configuración oculto}
\label{fig:menu_conf_oculto}
\end{figure}

\subsection{ABM de síntomas}
Para acceder a la pantalla de administración de síntomas nos dirigimos hacia ``Configuración'' y luego a ``Síntomas'' (ver figura \ref{fig:menu_sintomas}).
\begin{figure}
\centerline{\includegraphics[width=0.7\textwidth]{menu_sintomas.png}}
\caption{Menú de síntomas}
\label{fig:menu_sintomas}
\end{figure}
Allí se nos muestra el listado de todos los síntomas cargados en el sistema (figura \ref{fig:listado_sintomas}).

\begin{figure}
\centerline{\includegraphics[width=1\textwidth]{listado_sintomas.png}}
\caption{Listado de síntomas}
\label{fig:listado_sintomas}
\end{figure}

\subsubsection{Alta de síntoma}
Para dar de alta un nuevo síntoma hacemos click en el botón ``Nuevo'', en la pantalla del listado de síntomas, que nos dirige a la pantalla del detalle del síntoma (figura \ref{fig:detalle_sintoma}).
\begin{figure}
\centerline{\includegraphics[width=1\textwidth]{detalle_sintoma.png}}
\caption{Detalle de síntoma}
\label{fig:detalle_sintoma}
\end{figure}
Allí debemos ingresar el nombre, el discriminante, la prioridad para adultos y la prioridad para pediátricos. Tener en cuenta que si el discriminante que deseamos no aparece en el listado desplegable entonces primero debemos ingresarlo en el ABM de discriminantes (Ver sección \ref{ABM_discriminantes}). Luego de llenar todos los campos, el botón ``Aceptar'' se desbloquea y si le hacemos click aparece un mensaje que confirma que el síntoma fue ingresado con éxito (figura \ref{fig:sintoma_cargado_con_exito}).
\begin{figure}
\centerline{\includegraphics[width=1\textwidth]{sintoma_cargado_con_exito.png}}
\caption{Mensaje de síntoma cargado con éxito}
\label{fig:sintoma_cargado_con_exito}
\end{figure}

\subsubsection{Modificación de síntoma}
Para modificar un síntoma que hayamos dado de alta con anterioridad debemos seleccionarlo del listado de síntomas. Para facilitar la búsqueda del mismo podemos filtrar el listado llenando los campos de ``Síntoma'' y/o ``Discriminante''\footnote{El filtrado de los listados de síntomas, discriminantes y usuarios funciona de la misma manera. Por lo tanto en este manual sólo se explicará el filtrado del listado de síntomas.}. No hace falta que ingresemos la palabra entera. Es suficiente con ingresar las primeras letras. Tampoco se distinguen mayúsculas y minúsculas. Luego de ingresar el valor deseado hacemos click en el botón ``Buscar'' y si encontramos el síntoma buscado hacemos click en el botón ``Ver detalle''(ver figura \ref{fig:sintomas_filtro}) 
\begin{figure}
\centerline{\includegraphics[width=1\textwidth]{sintomas_listado_buscar.png}}
\caption{Listado de síntomas filtrado. Aparecen los botones ``Buscar'' y ``Ver detalle'' señalados en rojo}
\label{fig:sintomas_filtro}
\end{figure}
que nos lleva a la pantalla de ``Detalle de síntoma'' con todos los campos cargados. Allí podemos modificar los valores que deseemos y al apretar ``Aceptar'' aparece el mensaje de confirmación de síntoma actualizado con éxito (figura \ref{fig:sintoma_actualizado_con_exito}).
\begin{figure}
\centerline{\includegraphics[width=1\textwidth]{sintoma_actualizado_con_exito.png}}
\caption{Mensaje de confirmación de síntoma actualizado con éxito}
\label{fig:sintoma_actualizado_con_exito}
\end{figure}

\subsubsection{Baja de síntoma}
Una vez creados, los síntomas no se pueden eliminar. Solo se pueden modificar como explicamos en la sección anterior. Lo mismo sucede con los discriminantes de síntomas.

\subsection{ABM de discriminantes de síntomas}\label{ABM_discriminantes}
Para acceder a la pantalla de administración de discriminantes de síntomas nos dirigimos hacia ``Configuración'' y luego a ``Discriminantes'' (ver figura \ref{fig:menu_discriminantes}).
\begin{figure}
\centerline{\includegraphics[width=0.7\textwidth]{menu_discriminantes.png}}
\caption{Menú de discriminantes de síntomas}
\label{fig:menu_discriminantes}
\end{figure}
Allí se nos muestra el listado de todos los discriminantes de síntomas cargados en el sistema (figura \ref{fig:listado_discriminantes}).
\begin{figure}
\centerline{\includegraphics[width=1\textwidth]{listado_discriminantes.png}}
\caption{Listado de discriminantes}
\label{fig:listado_discriminantes}
\end{figure}








\newpage


 


\end{document}
