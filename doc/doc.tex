\documentclass[a4paper,10pt]{article}
\usepackage[utf8]{inputenc}
\usepackage[spanish]{babel}

%opening
\title{Trabajo de Inserción Profesional}
\author{Muñoz Nestor \\ Tejeda Marcia \\ \\ Director: Nicolás Paez \\  Co-Director: Pablo E. Martínez López}



\begin{document}



\maketitle
\newpage 
\begin{abstract}
Acá vamos a resumir todo el documento
\end{abstract}


\newpage 
\tableofcontents


\newpage 
\section{Introducción}
\subsection{Contexto general}
\subsection{Sobre el TRIAGE}
Acá contamos qué es el Triage

\subsection{Propuesta de solución}
Cómo vamos a encarar el problema
\subsection{Objetivo General}
Cómo queremos que sea el resultado
\subsection{Resultado Final}
Cuál fue el resultado
\subsection{Síntesis de trabajo}

\newpage 
\section{Planteo}
\subsection{Problema en detalle}
\subsection{Reuniones con el usuario}
\subsection{Diseño}

\subsection{Requerimientos del cliente}
\subsection{Pantallas}
Contamos en general cuál sería la dinámica de uso y las pantallas principales
\subsection{Informes}
Qué informes nos pidió el cliente y para qué

\subsubsection{Tiempo de espera para cada prioridad}
Consultar a Luis para qué necesita este reporte
\subsubsection{Cantidad de atenciones para cada prioridad}
Consultar a Luis para qué necesita este reporte
\subsubsection{Reporte de Personas}
Lista con todas las personas que se atendieron y el detalle de cada atención. Preguntar a Luis por qué es útil.


\newpage 
\section{Diseño e implementación}
\subsection{Tecnologías}
Acá contamos la elección de las tecnologías

\subsection{Metodología de trabajo}
Cuáles fueron las metodologías de trabajo y qué herramientas utilizamos

\subsection{Relevamiento}

\subsection{Diseño y arquitectura}
Cómo pensamos el modelo: particularidades de persona y paciente; pantalla de espera (poder acceder desde cualquier pc a los pacientes)

\subsection{Detalle técnico}
División entre angular y grails

\subsection{Problemas que tuvimos}


\subsection{Detalles interesantes del código}
En el caso de que los haya 




\subsection{Testing}
Herramientas que usamos para testear, qué tipo de testing hicimos (funcional, unitario, de integración...)
\subsubsection{Funcional}
Selenium..
\subsubsection{Unitario}

\subsubsection{Integración}

\subsection{Deploy}


\subsubsection{Instalación}
Cómo realizamos la instalación en el entorno de trabajo donde se usará el producto
\subsubsection{Manual de usuario}
Se confeccionó un manual de usuario...
\subsubsection{Aprendizaje del usuario}
Cuánto tiempo llevó explicar el sistema, si fue fácil de entender..

\newpage 
\section{Conclusiones}
\subsection{Cómo atacamos el problema}
Contamos como siempre estuvimos en contacto con el cliente, validando cada pantalla y cada funcionalidad
\subsection{Sobre lo que sabíamos y lo que no}
\subsection{Sobre la necesidad del codirector o "nexo" con el cliente}




\newpage 

\begin{thebibliography}{3} 
\bibitem{} Derlet R, Kinser D, Lou R, et al. Prospective identification and triage of nonemergency patients out of an Emergency Department: a 5 years study. Ann Emerg Med 1996; 25:215-223.
\bibitem{} Manual de procedimiento. Recepción,  Acogida y Clasificación.  MSPBS. Paraguay 2011.
\bibitem{} Shore J, Warden S, The Art of Agile Development, O’Reilly Media, 2007.

\end{thebibliography}
 


\end{document}
