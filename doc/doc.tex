\documentclass[a4paper,10pt]{article}
\usepackage[utf8]{inputenc}
\usepackage[spanish]{babel}

%opening
\title{Trabajo de Inserción Profesional}
\author{Néstor Muñoz\\ Marcia Tejeda\\ \\ Director: Nicolás Paez \\  Co-Director: Pablo E. Martínez López}



\begin{document}



\maketitle
\newpage 
\begin{abstract}
El presente Trabajo de Inserción Profesional (TIP) se desarrolló en el contexto del desarrollo de un sistema para resolver necesidades de la guardia del Hospital Oñativia en Rafael Calzada.\\ 
El problema presentado en este TIP es automatizar la recepción en la guardia del Hospital utilizando el método de emergencias conocido como Triage.\\ 
La solución propuesta es el desarrollo de una aplicación web que sea accesible desde todos los puntos de atención de las guardias y que permita evaluar los casos que se presentan de manera eficiente y con los mismos parámetros.

\end{abstract}


\newpage 
\tableofcontents


\newpage 
\section{Introducción}
\subsection{Contexto general}
Actualmente la guardia del H.Z.G.A ''Dr. Arturo Oñativia'' de la localidad de Rafael Calzada, a cargo del Doctor Luis Reggiani, utiliza el método Triage \cite{Derlet} \cite{Manual} para la clasificación de pacientes según los síntomas que presenten. Hasta el momento todo el proceso se hace en forma manual, lo que implica algunos contratiempos:
\begin{itemize}
\item Depender de una persona (o varias) con todo el conocimiento.
\item Emplear demasiado tiempo para guardar datos y recolectarlos.
\item Obtener diferentes resultados (algunas veces incorrectos), pues diferentes personas usan a veces criterios diferentes para la toma de decisiones.
\end{itemize}
Según Luis Reggiani informatizar el proceso de Triage implicaría una mejora notable en el desempeño de la guardia. Se lograría una estandarización en la clasificación de síntomas, se agilizaría el ingreso y la obtención de datos de pacientes, se mejoraría la atención en general y se distinguirían de una manera más eficaz aquellos pacientes que necesiten una atención inmediata.
\subsection{Sobre el TRIAGE}
Triage es un método de la medicina de emergencias y desastres para la selección y clasificación de los pacientes basándose en las prioridades de atención, privilegiando la posibilidad de supervivencia, de acuerdo a las necesidades terapéuticas y los recursos disponibles. Trata de evitar que se retrase la atención del paciente que empeoraría su pronóstico por la demora en su atención. Un nivel que implique que el paciente puede ser demorado no quiere decir que el diagnóstico final no pueda ser una enfermedad grave, ya que un cáncer, por ejemplo, puede tener funciones vitales estables que no lleve a ser visto con premura. El triage prioriza el compromiso vital inmediato y las posibles complicaciones.
\subsection{Propuesta de solución}
Proponemos desarrollar y poner en funcionamiento un sistema informático que dé soporte al proceso de Triage en la guardia del H.Z.G.A ''Dr. Arturo Oñativia'' de la localidad de Rafael Calzada.\\
Dado que el sistema podría incluir muchísimas funcionalidades y al mismo tiempo existe una especificación detallada de los requerimientos, planteamos el proyecto con alcance variable con el compromiso de entrega de un software que resuelva la parte central del proceso de Triage. La idea es que el sistema desarrollado en el contexto de este trabajo sea puesto en marcha y utilizado por la institución promotora.
\subsection{Objetivo General}
Concretamente el sistema debe cubrir las siguientes funcionalidades mínimas:
\begin{itemize}
\item Recepción de pacientes mediante búsqueda de aquellos que ya fueron atendidos en el hospital e ingreso de los que se atienden por primera vez.
\item Toma de impresión visual inicial del paciente.
\item Toma de los signos vitales que presenta el paciente: presión arterial (sístole y diástole), frecuencia cardíaca, saturación de O2, frecuencia respiratoria, temperatura y glucosa.
\item Ingreso de los síntomas que presenta el paciente.
\item División de los síntomas por categorías (discriminantes) y asociación de prioridades a los mismos.
\item Lógica variada para los síntomas, según se trate de un paciente adulto o pediátrico, tanto para los valores de los signos vitales como para las prioridades de los síntomas.
\item Emisión de alerta al momento de detectarse un síntoma de prioridad uno, para que se ingrese al paciente de inmediato al shock room.
\item Posibilidad de extraer reportes de cantidad de consultas realizadas según prioridad y promedio de tiempo de espera de atención según prioridad.
\item Puesta en funcionamiento en cada sala de recepción de pacientes de guardia.
\end{itemize}
\subsection{Resultado Final}
Desarrollamos una aplicación web que cubre todas las funcionalidades mínimas detalladas anteriormente y además realiza las siguientes:
\begin{itemize}
\item Generación de reportes por paciente a modo de historial de atenciones en guardia con detalle de fecha, síntomas presentados, signos vitales, prioridad asignada y tipo de atención recibida.
\item Diferenciación entre usuarios administradores del sistema y usuarios comunes.
\item Posibilidad de detallar el tipo de atención recibida por el paciente luego de pasar por el proceso de Triage
\item Alta, baja y modificación de pacientes, síntomas, discriminantes de síntomas y usuarios del sistema.
\end{itemize}
\subsection{Síntesis de trabajo}
TODO

\newpage 
\section{Planteo}

El presente trabajo comenzó con el relevamiento de información mediante reuniones con el Dr.\ Reggiani, con quien hubo contacto constante durante todo el desarrollo. Las primeras reuniones fueron para describir el problema y las necesidades reales. Luego, durante la etapa de desarrollo, cada pantalla y funcionalidad fue validada por el usuario, con el propósito de llegar a un producto que fuera útil y consistente. A medida que avanzamos con el producto, se fue negociando el alcance agregando o quitando funcionalidades dependiendo del tiempo disponible y consultando con el Dr.\ la prioridad para cada tarea.

En esta sección iremos presentando los resultados de estos encuentros.


\subsection{Dinámica de trabajo en el hospital}
La guardia del H.Z.G.A ``Dr.\ Arturo Oñativia'' opera recibiendo a los pacientes en dos sectores: Pediatría y Adultos. Cada sector tiene definidos sus parámetros de evaluación de pacientes, pudiendo un síntoma tener una prioridad para los adultos y otra para los pacientes pediátricos. Hay una división entre los pacientes pediátricos también dependiendo de la edad, diferenciando bebés de meses y niños más grandes.

Al llegar a la guardia los pacientes son recibidos por el enfermero de guardia, quién utilizando el sistema desarrollado toma una impresión visual del paciente. Luego se pasa a la sala de toma de signos vitales donde otro enfermero controla la presión, glucosa en sangre, entre otros, y graba en el sistema los síntomas que presenta el paciente. En caso de encontrar algún síntoma de prioridad UNO (peligro de muerte o daño permanente) en alguna de las tres instancias mencionadas antes (Impresión Visual, Signos Vitales o Síntomas), el sistema deriva al paciente de inmediato a la sala de Shock.  

Una vez conocida la prioridad del paciente ingresado, hay tres caminos: 
\begin{itemize}
\item Atención Inmediata.
\item Atención dentro de los próximos 30 minutos. 
\item Atención en Consultorios Externos.
\end{itemize}
Una vez atendido el paciente, se termina el ciclo dentro del sistema; esto es, el sistema no guarda información post-triage. 



\subsection{Requerimientos del cliente}
La característica más mencionada por el Dr.\ fue, en las primeras entrevistas, registrar adecuadamente a los pacientes que ingresan. Para ello, se decidió guardar todos los datos de las personas (tal como Nombre, Apellido, Teléfono, DNI, Dirección, entre otros) para poder contar con una base de datos de todas las personas atendidas en caso de necesitarla. 

Otra de las prioridades detectadas fue la necesidad de completar el Triage de forma eficiente y con una respuesta rápida ante casos de urgencias. El cliente solicitó de manera excluyente que el sistema debia cortar cualquier interacción en el momento de detectar un caso de Prioridad UNO, para poder actuar con el apremio necesario.

Se detallan en secciones futuras los reportes que el cliente requerió.


\subsection{Pantallas y dinámica de uso}
Las pantallas, como fue mencionado se fueron diseñando y validando con el cliente en las reuniones periódicas.

La pantalla inicial (y principal) del sistema permite buscar a los pacientes por nombre, apellido, DNI o fecha de nacimiento. En el caso de que ya hayan sido atendidos en algún momento en la guardia, serán encontrados por el buscador y se podrá proceder a completar los datos del Triage. En caso de no encontrarlos, la misma pantalla permite ingresarlos al sistema en el momento generando un nuevo registro de una persona. 

La pantalla de Triage está divida en tres: Impresión Visual, Síntomas y Signos Vitales. Tiene una navegación definida por pestañas que permite navegar de forma fluida entre los tres formularios. Una vez que se cargan los datos deseados, el paciente pasa a una ``Lista de espera'', otra pantalla que permite ver qué pacientes están esperando atención. Permite también continuar el Triage; esto es, ingresar nuevamente a la pantalla de Triage y poder modificar o cargar nuevos síntomas. Esta característica es necesaria cuando la persona que toma la Impresión Visual está en un lugar físico distinto al del enfermero que toma los signos vitales, por ejemplo.

En el caso de que el paciente haya sido atendido, o derivado a consultorio externo, y se retire del hospital, el enfermero, o administrativo, ubicado en el puesto de salida debe buscar al paciente en la lista de espera y marcar la finalización de la atención con alguna de las opciones mencionadas anteriormente: Atención Inmediata, Atención dentro de los próximos 30 minutos o Atención en consultorios externos.

Entre las pantallas administrativas, o de configuración, se encuentran las de Alta y Modificación de:
\begin{itemize}
\item Síntomas
\item Discriminantes
\item Usuarios
\end{itemize}
%
permitiendo crear nuevos registros o modificar los existentes. En el caso de los usuarios, es posible también dar la baja.




\subsection{Informes}
El cliente pidió pantallas con los informes detallados a continuación.

\begin{description}
\item[Tiempo de espera para cada prioridad]  \mbox{} \\
Reporte que muestra el tiempo de espera medio para cada prioridad en un rango de tiempo dado por dos fechas. 


\item[Cantidad de atenciones para cada prioridad]\mbox{} \\
Reporte que muestra la cantidad de atenciones para cada prioridad en un rango de tiempo dado por dos fechas.

\item[Reporte de Personas] \mbox{} \\
Lista con todas las personas que se atendieron. Permite ver individualmente los datos de cada persona y una lista que muestra todas las veces que fue atendida, los síntomas presentados y el tipo de atención recibida.
\end{description}


\newpage 
\section{Diseño e implementación}
\subsection{Tecnologías}
Acá contamos la elección de las tecnologías

\subsection{Metodología de trabajo}
Cuáles fueron las metodologías de trabajo y qué herramientas utilizamos

\subsection{Relevamiento}
?

\subsection{Diseño y arquitectura}
Cómo pensamos el modelo: particularidades de persona y paciente; pantalla de espera (poder acceder desde cualquier pc a los pacientes).
Diagrama de clases 

\subsection{Detalle técnico}
División entre angular y grails

\subsection{Problemas que tuvimos}
?

\subsection{Detalles interesantes del código}
En el caso de que los haya 




\subsection{Testing}
Herramientas que usamos para testear, qué tipo de testing hicimos (funcional, unitario, de integración...)
\subsubsection{Funcional}
Selenium.. / Protractor / Casper
\subsubsection{Unitario}
Con Grails
\subsubsection{Integración}
Con Grails
\subsection{Deploy}
\subsubsection{Instalación}
Cómo realizamos la instalación en el entorno de trabajo donde se usará el producto
\subsubsection{Manual de usuario}
Se confeccionó un manual de usuario...
\subsubsection{Aprendizaje del usuario}
Cuánto tiempo llevó explicar el sistema, si fue fácil de entender..

\newpage 
\section{Conclusiones}
Cómo atacamos el problema\\
Contamos como siempre estuvimos en contacto con el cliente, validando cada pantalla y cada funcionalidad\\
Sobre lo que sabíamos y lo que no\\
Sobre la necesidad del codirector o 'nexo' con el cliente



\newpage 

\begin{thebibliography}{3} 
\bibitem{Derlet} Derlet R, Kinser D, Lou R, et al. Prospective identification and triage of nonemergency patients out of an Emergency Department: a 5 years study. Ann Emerg Med 1996; 25:215-223.
\bibitem{Manual} Manual de procedimiento. Recepción,  Acogida y Clasificación.  MSPBS. Paraguay 2011.
\bibitem{Shore} Shore J, Warden S, The Art of Agile Development, O’Reilly Media, 2007.

\end{thebibliography}
 


\end{document}
