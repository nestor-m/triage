\section{Planteo}
\subsection{Problema en detalle}
La guardia del H.Z.G.A ''Dr. Arturo Oñativia'' opera recibiendo a los pacientes en dos sectores: Pediatría y Adultos. \\
Cada sector tiene definidos sus parámetros de evaluación de pacientes, pudiendo un síntoma tener una prioridad para los adultos y otra para los pacientes pediátricos.\\
Hay una división entre los pacientes pediátricos también dependiendo de la edad, diferenciando bebés de meses y niños más grandes.\\
Al llegar a la guardia, los pacientes son recibidos por el enfermero de guardia que es quién utiliza el sistema desarrollado. Éste toma una impresión visual del paciente. Luego se pasa a la sala de toma de signos vitales donde un enfermero controla la presión, glucosa en sangre entre otros y graba en el sistema los síntomas que presenta el paciente. \\
En caso de encontrar algún síntoma de prioridad UNO en alguna de las tres instancias mencionadas antes (Impresión Visual, Signos Vitales o Síntomas), el sistema deriva al paciente de inmediato a la sala de Shock.  \\
Una vez conocida la prioridad del paciente ingresado, 



\subsection{Reuniones y contacto con el usuario}
Contacto constante (Esto aparece abajo también, no sé dónde iría o cómo dividirlo)
\subsection{Requerimientos del cliente}
AGREGO ACÁ
\subsection{Pantallas}
Contamos en general cuál sería la dinámica de uso y las pantallas principales
\subsection{Informes}
Qué informes nos pidió el cliente y para qué
\subsubsection{Tiempo de espera para cada prioridad}
Consultar a Luis para qué necesita este reporte
\subsubsection{Cantidad de atenciones para cada prioridad}
Consultar a Luis para qué necesita este reporte
\subsubsection{Reporte de Personas}
Lista con todas las personas que se atendieron y el detalle de cada atención. Preguntar a Luis por qué es útil