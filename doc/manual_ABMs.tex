\mySection{Alta, baja y modificación (ABM) de datos del sistema}
En esta sección explicamos como realizar el alta, baja y modificación de datos del sistema, es decir, síntomas, discriminantes y usuarios.

\mySubSection{Acceso al menú de configuración}
Para acceder al menú de configuración de datos, el usuario actual debe tener el rol de administrador (explicaremos la asignación de roles en la sección \ref{ABM_usuarios}). En caso contrario dicho menú permanecerá oculto (ver figuras \ref{fig:menu_conf_visible} y \ref{fig:menu_conf_oculto}).

\begin{figure}
\centerline{\includegraphics[width=0.7\textwidth]{menu_configuracion_visible.png}}
\caption{Menú de configuración visible}
\label{fig:menu_conf_visible}
\end{figure}

\begin{figure}
\centerline{\includegraphics[width=0.7\textwidth]{menu_configuracion_oculto.png}}
\caption{Menú de configuración oculto}
\label{fig:menu_conf_oculto}
\end{figure}

\mySubSection{ABM de síntomas}\label{cap:ABM_sintomas}
Para acceder a la pantalla de administración de síntomas nos dirigimos hacia ``Configuración'' y luego a ``Síntomas'' (ver figura \ref{fig:menu_sintomas}).
\begin{figure}
\centerline{\includegraphics[width=0.7\textwidth]{menu_sintomas.png}}
\caption{Menú de síntomas}
\label{fig:menu_sintomas}
\end{figure}
Allí se nos muestra el listado de todos los síntomas cargados en el sistema (figura \ref{fig:listado_sintomas}).

\begin{figure}
\centerline{\includegraphics[width=1\textwidth]{listado_sintomas.png}}
\caption{Listado de síntomas}
\label{fig:listado_sintomas}
\end{figure}

\begin{description}
\item[Alta de síntoma] \mbox{} \\
%\subsubsection{Alta de síntoma}\label{cap:alta_sintoma}
En la pantalla del listado de síntomas hacemos click en el botón ``Nuevo'' que nos dirige a la pantalla del detalle del síntoma (figura \ref{fig:detalle_sintoma}).
\begin{figure}
\centerline{\includegraphics[width=1\textwidth]{detalle_sintoma.png}}
\caption{Detalle de síntoma}
\label{fig:detalle_sintoma}
\end{figure}
Allí debemos ingresar el nombre, el discriminante, la prioridad para adultos y la prioridad para pediátricos. Tener en cuenta que si el discriminante que deseamos no aparece en el listado desplegable entonces primero debemos darle de alta (ver sección \ref{cap:ABM_discriminantes}). Luego de llenar todos los campos, el botón ``Aceptar'' se desbloquea y si le hacemos click debería aparecer un mensaje que confirma que el síntoma fue ingresado con éxito.

\item[Modificación de síntoma] \mbox{} \\
%\subsubsection{Modificación de síntoma}\label{cap:modificacion_sintoma}
Para modificar un síntoma debemos seleccionarlo del listado. Para facilitar la búsqueda del mismo podemos filtrar el listado llenando los campos de ``Síntoma'' y/o ``Discriminante'' (ver sección \ref{cap:filtrado_listado}). Si encontramos el registro buscado hacemos click en el botón ``Ver detalle''(ver figura \ref{fig:sintomas_filtro}) 
\begin{figure}
\centerline{\includegraphics[width=1\textwidth]{sintomas_listado_buscar.png}}
\caption{Listado de síntomas filtrado. Aparecen los botones ``Buscar'' y ``Ver detalle'' señalados en rojo}
\label{fig:sintomas_filtro}
\end{figure}
que nos lleva a la pantalla de ``Detalle de síntoma'' con todos los campos cargados. Allí podemos modificar los valores que deseemos y al presionar el botón ``Aceptar'' debería aparecer el mensaje de confirmación de síntoma actualizado con éxito.

\item[Baja de síntoma] \mbox{} \\
%\subsubsection{Baja de síntoma}
Una vez creados, los síntomas no se pueden eliminar. Solo se pueden modificar como explicamos en la sección anterior. Lo mismo sucede con los discriminantes.

\end{description}

\mySubSection{ABM de discriminantes de síntomas}\label{cap:ABM_discriminantes}
Para acceder a la pantalla de administración de discriminantes nos dirigimos hacia ``Configuración'' y luego a ``Discriminantes'' (ver figura \ref{fig:menu_discriminantes}).
\begin{figure}
\centerline{\includegraphics[width=0.7\textwidth]{menu_discriminantes.png}}
\caption{Menú de discriminantes de síntomas}
\label{fig:menu_discriminantes}
\end{figure}
Allí se nos muestra el listado de todos los discriminantes cargados en el sistema (figura \ref{fig:listado_discriminantes}).
\begin{figure}
\centerline{\includegraphics[width=1\textwidth]{listado_discriminantes.png}}
\caption{Listado de discriminantes de síntomas}
\label{fig:listado_discriminantes}
\end{figure}

\begin{description}
\item[Alta de discriminante de síntoma] \mbox{} \\
%\subsubsection{Alta de discriminante de síntoma}\label{cap:alta_discriminante}
En la pantalla del listado de discriminantes hacemos click en el botón ``Nuevo'' que nos dirige a la pantalla del detalle del discriminante (figura \ref{fig:nuevo_discriminante}).
\begin{figure}
\centerline{\includegraphics[width=1\textwidth]{nuevo_discriminante.png}}
\caption{Alta de discriminante de síntoma}
\label{fig:nuevo_discriminante}
\end{figure}
Allí debemos ingresar el nombre, lo que desbloquea el botón ``Aceptar'' y si le hacemos click debería aparecer un mensaje que confirma que el discriminante fue ingresado con éxito. Una vez cargado podremos ingresar nuevos síntomas de ese discriminante, como explicamos anteriormente en la sección \ref{cap:ABM_sintomas}.

\item[Modificación de un discriminante de síntoma] \mbox{} \\
%\subsubsection{Modificación de un discriminante de síntoma}
Para modificar el nombre de un discriminante debemos seleccionarlo del listado\footnote{Por cuestiones de lógica del proceso de Triage podemos modificar el nombre de cualquier discriminante a excepción de ``IMPRESIÓN INICIAL''.}. Para facilitar la búsqueda del mismo podemos filtrar el listado llenando el campo ``Discriminante'' (ver sección \ref{cap:filtrado_listado}). Si encontramos el registro buscado hacemos click en el botón ``Ver detalle'' que nos lleva a la pantalla de ``Detalle de discriminante'' con el campo ``Nombre'' cargado y con un listado que nos muestra todos los síntomas de ese discriminante (ver figura \ref{fig:detalle_discriminante}).
\begin{figure}
\centerline{\includegraphics[width=1\textwidth]{listado_sintomas_de_discriminante.png}}
\caption{Detalle del discriminante con listado de síntomas}
\label{fig:detalle_discriminante}
\end{figure}
En esa pantalla podemos modificar el nombre y al presionar el botón ``Aceptar'' debería aparecer el mensaje de confirmación de discriminante actualizado con éxito. También podemos hacer click en el botón ``Ver detalle'' de algún síntoma del listado para ir a la pantalla de ``Detalle de síntoma'' y modificarlo como mostramos anteriormente en la sección \ref{cap:ABM_sintomas}.

\item[Baja de discriminante de síntoma] \mbox{} \\
%\subsubsection{Baja de discriminante de síntoma}
Al igual que los síntomas, una vez creados, los discriminantes no se pueden eliminar. Solo podemos modificarles el nombre como explicamos en la sección anterior.
\end{description}

\mySubSection{ABM de usuarios}\label{ABM_usuarios}
Para acceder a la pantalla de administración de usuarios nos dirigimos hacia ``Configuración'' y luego a ``Usuarios'' (ver figura \ref{fig:menu_usuarios}).
\begin{figure}
\centerline{\includegraphics[width=0.7\textwidth]{menu_usuarios.png}}
\caption{Menú de usuarios}
\label{fig:menu_usuarios}
\end{figure}
Allí se nos muestra el listado de todos los usuarios cargados en el sistema (figura \ref{fig:listado_usuarios}).
\begin{figure}
\centerline{\includegraphics[width=1\textwidth]{listado_usuarios.png}}
\caption{Listado de usuarios}
\label{fig:listado_usuarios}
\end{figure}

\begin{description}
\item[Roles] \mbox{} \\
%\subsubsection{Roles}\label{cap:roles}
Hay dos roles: ``administrador'' y ``usuario''. El usuario con rol ``administrador'' puede ingresar en todas las pantallas de la aplicación. En cambio, el usuario con rol ``usuario'' puede ingresar en todas las pantallas excepto en la de configuración.

\item[Alta de usuario] \mbox{} \\
%\subsubsection{Alta de usuario}\label{cap:alta_usuario}
En la pantalla del listado de usuarios hacemos click en el botón ``Nuevo'' que nos dirige a la pantalla del detalle del usuario (figura \ref{fig:nuevo_usuario}).
\begin{figure}
\centerline{\includegraphics[width=1\textwidth]{nuevo_usuario.png}}
\caption{Alta de usuario}
\label{fig:nuevo_usuario}
\end{figure}
Allí ingresamos el nombre y el rol, y luego presionamos el botón ``Aceptar'' (que desbloqueamos al llenar todos los campos). Nos debería aparecer un mensaje que confirma que el usuario fue ingresado con éxito. Una vez cargado podremos ingresar a la aplicación con ese usuario. Tener en cuenta que todos los usuarios son creados con la contraseña ``triage" (anteriormente explicamos cómo cambiarla en la sección \ref{cap:cambio_pass}).

\item[Modificación de usuario] \mbox{} \\
%\subsubsection{Modificación de un usuario}
Para modificar el nombre o el rol de un usuario debemos seleccionarlo del listado. Para facilitar la búsqueda del mismo podemos filtrar el listado llenando el campo ``Usuario'' (ver sección \ref{cap:filtrado_listado}). Si encontramos el registro buscado hacemos click en el botón ``Ver detalle'' que nos lleva a la pantalla de ``Detalle de usuario'' con todos los campos cargados. Allí podemos modificar los valores que deseemos y al presionar el botón ``Aceptar'' debería aparecer el mensaje de confirmación de usuario actualizado con éxito. Tener en cuenta que el usuario modificado será deslogueado.

\item[Baja de usuario] \mbox{} \\
Para dar de baja un usuario debemos presionar el botón ``Eliminar'' del listado. Si confirmamos la acción deberia aparecer el mensaje de ``Usuario eliminado con éxito'' (figura \ref{fig:eliminar_usuario}).
\begin{figure}
\centerline{\includegraphics[width=1\textwidth]{eliminar_usuario.png}}
\caption{Baja de usuario}
\label{fig:eliminar_usuario}
\end{figure}
El usuario será deslogueado y ya no podrá ingresar a la aplicación. Tener en cuenta que el sistema no permite que se eliminen todos los usuarios administradores.

\end{description}