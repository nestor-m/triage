\documentclass[a4paper,10pt]{article}
\usepackage[utf8]{inputenc}
\usepackage[spanish]{babel}
%links en el indice
\usepackage[bookmarks = true, colorlinks=true, linkcolor = black, citecolor = black, menucolor = black, urlcolor = black]{hyperref} 
\usepackage{graphicx}



%opening
\title{Triage, Sistema de gestión para sala de Guardia Hospitalaria}
\author{\textit{Manual del usuario}}

\begin{document}


\maketitle
\newpage 

\begin{abstract}
El presente documento tiene como objetivo describir y explicar las funcionalidad del Sistema de gestión para sala de Guardia Hospitalaria, que utiliza el método Triage para la recepción de los pacientes.\\
En el mismo, se detalla la funcionalidad de cada pantalla con screenshot y ejemplos básicos y funcionales al manual.\\

\end{abstract}


\newpage 
\tableofcontents

\newpage

\mySection{Pantalla Inicial}
En esta sección del manual explicaremos cómo ingresar nuevas personas al sistema o cómo buscarlas en el caso de que ya hayan sido atendidas.

Ingresar una nueva persona a la aplicación significa que sus datos quedarán guardados en el sistema, para poder realizar consultas sobre las atenciones recibidas o bien, para no tener que volver a cargar los datos si el paciente vuelve a atenderse en otro momento.

\mySubSection{Ingreso de un nuevo paciente}
En la pantalla de Inicio del sistema (figura \ref{fig:inicio}) 
\begin{figure}
\centerline{\includegraphics[width=0.99\textwidth]{inicio.png}}
\caption{Pantalla inicial} \label{fig:inicio}
\end{figure}
se pueden ver los campos identitificatorios de las personas. El botón ``Ingresar nuevo paciente'' aparecerá deshabilitado hasta completar los campos obligatorios: DNI, nombre, apellido y fecha de nacimiento (como puede verse en la figura \ref{fig:inicio_nuevo}).
\begin{figure}
\centerline{\includegraphics[width=0.99\textwidth]{inicio_nuevo.png}}
\caption{Botón habilitado para poder cargar nuevo paciente} \label{fig:inicio_nuevo}
\end{figure}
Al presionar dicho botón, el sistema guardará los datos de esa persona y redigirá la aplicación a la ventana de Triage para comenzar con la carga de síntomas.

\mySubSection{Búsqueda e ingreso de un paciente cargado en sistema}
En el caso de que el paciente ya haya recibido atención en la guardia, es posible buscarlo en la aplicación. El botón ``buscar'' apacerá deshabilitado hasta que se complete alguno de los campos (figura \ref{fig:inicio_busqueda}).
\begin{figure}
\centerline{\includegraphics[width=0.99\textwidth]{inicio_busqueda.png}}
\caption{Botón habilitado para poder buscar un paciente y botón para ingresar al paciente a Triage} \label{fig:inicio_busqueda}
\end{figure}
Una vez presionado el botón para buscar, el sistema muestra en el listado inferior la lista de personas que coinciden con los criterios de búsqueda ingresados (ver sección \ref{cap:filtrado_listado}). En el caso de ver a la persona que se está buscando, en la lista aparece el botón ``Ingresar''. Al presionarlo, el sistema redirige la aplicación a la ventana de Triage para comenzar con la carga de síntomas.

\mySubSection{Filtrado de un listado}\label{cap:filtrado_listado}
Todos los listados de la aplicación pueden ser filtrados para facilitar la búsqueda de algún registro. El modo de filtrado es muy sencillo. Los pasos a seguir son los siguientes:
\begin{enumerate}
\item ingresamos algun texto en el/los campo/s de búsqueda (no hace falta que ingresemos la palabra entera de lo que buscamos, es suficiente si solo ingresamos las primeras letras).
\item presionamos el boton ``Buscar''.
\end{enumerate}

 

\section{Triage}
En esta sección daremos a conocer el camino que recorre la aplicación para cargar los síntomas del paciente.

\subsection{Pantalla inicial de Triage}
La pantalla inicial de Triage (como podemos ver en la figura \ref{fig:triage_inicial}) 
\begin{figure}
\centerline{\includegraphics[width=0.99\textwidth]{impresion_visual.png}}
\caption{Pantalla inicial de Triage} \label{fig:triage_inicial}
\end{figure}
tiene una navegación definida por pestañas (que se pueden ver en la parte izquierda de la pantalla). Las pestañas permiten cambiar de pantalla de manera rápida y simple.

\subsection{Impresión Visual}
En la pestaña de impresión visual (figura \ref{fig:triage_inicial}) se cargan los síntomas que el enfermero/administrativo ve en el paciente que está siendo atendido. Una vez seleccionados los síntomas visuales, el usuario debe presionar el botón ``Aceptar''. En el caso de que un síntoma de Prioridad UNO sea seleccionado el sistema corta la interacción con el usuario con un cartel de confirmación (figura \ref{fig:impresion_visual_p1}).
\begin{figure}
\centerline{\includegraphics[width=0.99\textwidth]{impresion_visual_p1.png}}
\caption{Pantalla inicial de Triage} \label{fig:impresion_visual_p1}
\end{figure}
Si el usuario confirma, el sistema deriva directamente a la pantalla de Prioridad UNO, mostrando los datos y síntomas del paciente ingresado (figura \ref{fig:prioridad_uno}).
\begin{figure}
\centerline{\includegraphics[width=0.99\textwidth]{prioridad_uno.png}}
\caption{Prioridad UNO} \label{fig:prioridad_uno}
\end{figure}

\subsection{Síntomas}
En la pestaña de síntomas (figura \ref{fig:sintomas})
\begin{figure}
\centerline{\includegraphics[width=0.99\textwidth]{sintomas.png}}
\caption{Pestaña de síntomas} \label{fig:sintomas}
\end{figure}
van a ser cargados los síntomas que el paciente informe. 

En el cuadro central se pueden ver todos los síntomas cargados en sistema, indicando cuál es su discriminante. Aquí se puede filtrar también por síntoma o discriminante (tal como se explica en la sección 'Filtrado del listado') (Ver figura \ref{fig:sintomas_filtrar}).
\begin{figure}
\centerline{\includegraphics[width=0.99\textwidth]{sintomas_buscar.png}}
\caption{Filtrado en el cuadro de síntomas} \label{fig:sintomas_filtrar}
\end{figure}
Una vez filtrado el listado y encontrado lo que se busca, en cada fila del cuadro se puede ver el botón ``Agregar'' (figura \ref{fig:sintomas_agregar}), que permite cargar un nuevo síntoma al paciente.
\begin{figure}
\centerline{\includegraphics[width=0.99\textwidth]{sintomas_agregar.png}}
\caption{Agregar nuevo síntoma} \label{fig:sintomas_agregar}
\end{figure}
En la parte derecha de la pantalla se pueden ver los síntomas ya cargados. Se puede también eliminar algún síntoma agregado mediante el botón ``Borrar'' (que aparece al pararse con el puntero sobre el elemento a eliminar).

En el caso de ingresar un síntoma con Prioridad UNO,  el sistema corta la interacción con el usuario con un cartel de confirmación. Si el usuario confirma que efectivamente ese es el síntoma a agregar, el sistema deriva directamente a la pantalla de Prioridad UNO, mostrando los datos y síntomas del paciente ingresado (figura \ref{fig:prioridad_uno}).

Al finalizar con la carga, se debe presionar el botón ``Aceptar'' para grabar los síntomas seleccionados.


\subsection{Signos Vitales}
La tercer pestaña del Triage es para completar los signos vitales del paciente (figura \ref{fig:signos_vitales}).
\begin{figure}
\centerline{\includegraphics[width=0.99\textwidth]{signos_vitales.png}}
\caption{Signos Vitales} \label{fig:signos_vitales}
\end{figure}
Cada signo vital está definido para ser seleccionado de una lista acotada. En el caso de seleccionar algún valor que corresponda a una Prioridad UNO, el sistema mostrará un mensaje de confirmación. Si el usuario confirma la acción, se corta toda interacción mostrando la pantalla que indica atención inmediata (figura \ref{fig:prioridad_uno}).

Al finalizar de cargar los signos vitales, se debe presionar el botón ``Guardar'' (figura \ref{fig:signos_vitales_guardar})
\begin{figure}
\centerline{\includegraphics[width=0.99\textwidth]{signos_vitales_guardar.png}}
\caption{Signos Vitales} \label{fig:signos_vitales_guardar}
\end{figure}
y el sistema informará que los datos se han guardado con éxito.


\subsection{Fin de la carga}
Al terminar de cargar los síntomas hay dos caminos:
\begin{description}
\item[Finalizar Triage]  \mbox{} \\
Para finalizar el Triage, se debe presionar el botón sobre la pestaña izquieda llamado ``Fin Triage'' (figura \ref{fin_triage}).
\begin{figure}
\centerline{\includegraphics[width=0.99\textwidth]{fin_triage.png}}
\caption{Signos Vitales} \label{fig:fin_triage}
\end{figure}
 Al hacer esto, el sistema calcula la prioridad del paciente, indicando todos los síntomas cargados y sus datos personales. 

Esta acción sólo puede mostrar las pantallas de Prioridad DOS (figura \ref{fig:prioridad_dos}) 
\begin{figure}
\centerline{\includegraphics[width=0.99\textwidth]{prioridad_dos.png}}
\caption{Prioridad DOS} \label{fig:prioridad_dos}
\end{figure}
y Prioridad TRES (figura \ref{fig:prioridad_tres}), 
\begin{figure}
\centerline{\includegraphics[width=0.99\textwidth]{prioridad_tres.png}}
\caption{Signos Vitales} \label{fig:prioridad_tres}
\end{figure}
ya que la pantalla de Prioridad UNO sólo se presenta al seleccionar un síntoma de Prioridad UNO y corta toda interacción con el usuario.
Finalizar el Triage no quita al paciente de la lista de espera, simplemente calcula su prioridad. 

\item[Salir de la carga]\mbox{} \\
En el caso de querer abandonar la carga de síntomas para poder retomarla más tarde, el sistema provee la acción ``Salir'' (figura \ref{fin}),
\begin{figure}
\centerline{\includegraphics[width=0.99\textwidth]{fin.png}}
\caption{Signos Vitales} \label{fig:fin}
\end{figure}
que permite guardar los síntomas ingresados hasta el momento y poder recuperarlos si se carga el paciente desde la lista de espera.


\end{description}

\mySection{Pacientes en espera}
En la secciones anteriores se describió la manera de completar el Triage. 

Cada vez que se termina de cargar los síntomas hay dos caminos: ``Fin Triage'' o ``Salir''. Ambas opciones dejan al paciente en una ``Lista de espera'' (figura \ref{fig:espera}).
\begin{figure}
\centerline{\includegraphics[width=0.99\textwidth]{espera.png}}
\caption{Lista de espera} \label{fig:espera}
\end{figure}
 La pantalla de lista de espera contiene un cuadro como los vistos anteriormente de personas así como los filtros para realizar búsquedas en ese cuadro. El cuadro muestra, también, el tiempo que ha pasado desde que el paciente ingresó a la guardia.

Esta pantalla es muy útil cuando la carga de síntomas es realizada por dos personas en ubicaciones físicas distinas. El paciente puede ser atendido en un mostrador (donde se toman sus síntomas, por ejemplo), y luego pasar a un consultorio para la toma de signos vitales. El enfermero en el consultorio debe solamente buscar al paciente en la lista de espera y podrá continuar grabando sus síntomas sin perder ninguna información ya cargada.

\mySubSection{Continuar el Triage}
En el caso de querer continuar el Triage de un paciente en lista de espera, el sistema permite buscar al paciente utilizando el filtro. Una vez localizado, a nivel fila del cuadro encontramos la opción ``Triage'' (figura \ref{fig:espera1})
\begin{figure}
\centerline{\includegraphics[width=0.99\textwidth]{espera1.png}}
\caption{Continuar el Triage} \label{fig:espera1}
\end{figure}
que permite volver a la pantalla de Triage del paciente seleccionado para continuar cargando síntomas o volver a calcular la prioridad.

\mySubSection{Finalizar}
Para finalizar el Triage, esto es, que el paciente salga del hospital o entre a consultorios/internación para recibir atención, el cuadro muestra, a nivel fila, la opción ``Finalizar'' (figura \ref{fig:espera2})
\begin{figure}
\centerline{\includegraphics[width=0.99\textwidth]{espera2.png}}
\caption{Finalizar} \label{fig:espera2}
\end{figure}
que muestra una pantalla para indicar cuál fue la atención recibida por el paciente asi como sus datos y los síntomas cargados (figura \ref{fig:fin_atencion}).
\begin{figure}
\centerline{\includegraphics[width=0.99\textwidth]{fin_atencion.png}}
\caption{Fin de la atención} \label{fig:fin_atencion}
\end{figure}

Antes de finalizar la atención, el sistema guarda qué tipo de atención recibió el paciente, las cuales son:

\begin{itemize}
\item Ingresa para atención
\item Consultorio externo
\item Se retira sin atención
\end{itemize}

\section{Alta, baja y modificación de datos (ABMs)}
En esta sección explicamos como realizar el alta, baja y modificación de datos del sistema, es decir, síntomas, discriminantes y usuarios.

\subsection{Acceso al menú de configuración}
Para poder acceder al menú de configuración de datos el usuario actual debe tener el rol de administrador. En caso contrario dicho menú permanecerá oculto (ver figuras \ref{fig:menu_conf_visible} y \ref{fig:menu_conf_oculto}).

\begin{figure}
\centerline{\includegraphics[width=0.7\textwidth]{menu_configuracion_visible.png}}
\caption{Menú de configuración visible}
\label{fig:menu_conf_visible}
\end{figure}

\begin{figure}
\centerline{\includegraphics[width=0.7\textwidth]{menu_configuracion_oculto.png}}
\caption{Menú de configuración oculto}
\label{fig:menu_conf_oculto}
\end{figure}

\subsection{ABM de síntomas}
Para acceder a la pantalla de administración de síntomas nos dirigimos hacia ``Configuración'' y luego a ``Síntomas'' (ver figura \ref{fig:menu_sintomas}).
\begin{figure}
\centerline{\includegraphics[width=0.7\textwidth]{menu_sintomas.png}}
\caption{Menú de síntomas}
\label{fig:menu_sintomas}
\end{figure}
Allí se nos muestra el listado de todos los síntomas cargados en el sistema (figura \ref{fig:listado_sintomas}).

\begin{figure}
\centerline{\includegraphics[width=1\textwidth]{listado_sintomas.png}}
\caption{Listado de síntomas}
\label{fig:listado_sintomas}
\end{figure}

\subsubsection{Alta de síntoma}
Para dar de alta un nuevo síntoma hacemos click en el botón ``Nuevo'', en la pantalla del listado de síntomas, que nos dirige a la pantalla del detalle del síntoma (figura \ref{fig:detalle_sintoma}).
\begin{figure}
\centerline{\includegraphics[width=1\textwidth]{detalle_sintoma.png}}
\caption{Detalle de síntoma}
\label{fig:detalle_sintoma}
\end{figure}
Allí debemos ingresar el nombre, el discriminante, la prioridad para adultos y la prioridad para pediátricos. Tener en cuenta que si el discriminante que deseamos no aparece en el listado desplegable entonces primero debemos ingresarlo en el ABM de discriminantes (Ver sección \ref{ABM_discriminantes}). Luego de llenar todos los campos, el botón ``Aceptar'' se desbloquea y si le hacemos click aparece un mensaje que confirma que el síntoma fue ingresado con éxito (figura \ref{fig:sintoma_cargado_con_exito}).
\begin{figure}
\centerline{\includegraphics[width=1\textwidth]{sintoma_cargado_con_exito.png}}
\caption{Mensaje de síntoma cargado con éxito}
\label{fig:sintoma_cargado_con_exito}
\end{figure}

\subsubsection{Modificación de síntoma}
Para modificar un síntoma que hayamos dado de alta con anterioridad debemos seleccionarlo del listado de síntomas. Para facilitar la búsqueda del mismo podemos filtrar el listado llenando los campos de ``Síntoma'' y/o ``Discriminante''\footnote{El filtrado de los listados de síntomas, discriminantes y usuarios funciona de la misma manera. Por lo tanto en este manual sólo se explicará el filtrado del listado de síntomas.}. No hace falta que ingresemos la palabra entera. Es suficiente con ingresar las primeras letras. Tampoco se distinguen mayúsculas y minúsculas. Luego de ingresar el valor deseado hacemos click en el botón ``Buscar'' y si encontramos el síntoma buscado hacemos click en el botón ``Ver detalle''(ver figura \ref{fig:sintomas_filtro}) 
\begin{figure}
\centerline{\includegraphics[width=1\textwidth]{sintomas_listado_buscar.png}}
\caption{Listado de síntomas filtrado. Aparecen los botones ``Buscar'' y ``Ver detalle'' señalados en rojo}
\label{fig:sintomas_filtro}
\end{figure}
que nos lleva a la pantalla de ``Detalle de síntoma'' con todos los campos cargados. Allí podemos modificar los valores que deseemos y al apretar ``Aceptar'' aparece el mensaje de confirmación de síntoma actualizado con éxito (figura \ref{fig:sintoma_actualizado_con_exito}).
\begin{figure}
\centerline{\includegraphics[width=1\textwidth]{sintoma_actualizado_con_exito.png}}
\caption{Mensaje de confirmación de síntoma actualizado con éxito}
\label{fig:sintoma_actualizado_con_exito}
\end{figure}

\subsubsection{Baja de síntoma}
Una vez creados, los síntomas no se pueden eliminar. Solo se pueden modificar como explicamos en la sección anterior. Lo mismo sucede con los discriminantes de síntomas.

\subsection{ABM de discriminantes de síntomas}\label{ABM_discriminantes}
Para acceder a la pantalla de administración de discriminantes de síntomas nos dirigimos hacia ``Configuración'' y luego a ``Discriminantes'' (ver figura \ref{fig:menu_discriminantes}).
\begin{figure}
\centerline{\includegraphics[width=0.7\textwidth]{menu_discriminantes.png}}
\caption{Menú de discriminantes de síntomas}
\label{fig:menu_discriminantes}
\end{figure}
Allí se nos muestra el listado de todos los discriminantes de síntomas cargados en el sistema (figura \ref{fig:listado_discriminantes}).
\begin{figure}
\centerline{\includegraphics[width=1\textwidth]{listado_discriminantes.png}}
\caption{Listado de discriminantes}
\label{fig:listado_discriminantes}
\end{figure}







\end{document}