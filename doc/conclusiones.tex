\section{Conclusiones}


\paragraph{}
Desde un primer momento estuvimos en contacto constante con Luis, quien se encargó de validar cada funcionalidad y cada nueva pantalla a desarrollar. Gracias a esto y a los métodos ágiles utilizados durante el desarrollo, el margen de error fue muy pequeño, ya que cada semana se presentaba una nueva funcionalidad .\\
El trabajo en general fue un ida y vuelta de ideas. Contamos con el apoyo constante de nuestro Director Nicolás y nuestro Codirector Fidel. La buena voluntad del cliente facilitó la comunicación, evitando muchas veces trabajo en vano.\\

\paragraph{}
Fue fundamental durante todo el desarrollo tener en cuenta y seguir la idea de Luis: desarrollar una aplicación simple, sencilla y fácil de usar que resolviera el problema en cuestión.\\
El producto fue certificado por Luis luego de varias demostraciones y de haber dejado en el hospital una copia de la aplicación para realizar pruebas y familiarizarse con el entorno.\\

\paragraph{}


Sobre lo que aprendimos\\
Sobre lo que sabíamos y lo que no\\


\paragraph{}
En nuestro trabajo contamos con un Codirector, la motivación para incorporarlo se debe a que el profesor Pablo E. Martínez López tiene amplia experiencia en cuestiones de gestión, las cuales junto con el trabajo del director complementarán el conocimiento de ambos para un mejor desarrollo del trabajo de los alumnos.
