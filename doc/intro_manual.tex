\mySection{Introducción}
Las siguientes secciones tienen como objetivo describir y explicar las funcionalidades del Sistema de gestión para sala de Guardia Hospitalaria, que utiliza el método Triage para la recepción de los pacientes. En las mismas, se detalla la funcionalidad de cada pantalla con screenshot y ejemplos básicos y funcionales al manual.

Triage es un método de medicina de emergencias y desastres para la selección y clasificación de los pacientes basándose en las prioridades de atención. La guardia del H.Z.G.A ``Dr.\ Arturo Oñativia'' de la localidad de Rafael Calzada utiliza este sistema para clasificar a sus pacientes. El triage prioriza el compromiso vital inmediato y las posibles complicaciones.
Los pacientes pueden ser clasificados con tres Prioridades:

\begin{description}
\item[Prioridad 1] \mbox{} \\ 
Cuando el paciente tiene posibilidad de sobrevivir y la actuación médica debe ser inmediata.
\item[Prioridad 2] \mbox{} \\ 
Pacientes que presentan una situación de urgencia con riesgo vital.
\item[Prioridad 3] \mbox{} \\ 
Paciente levemente lesionado, que puede caminar y su traslado no precisa medio especial.

\end{description}

